\documentclass[a4paper,12pt]{article}
\usepackage[a4paper, margin=1in]{geometry} % Sets the paper size to A4 and margins to 1 inch
\usepackage{graphicx}
\usepackage{subcaption} % for subfigures
\usepackage{setspace}		
\usepackage{xcolor,listings} % for typesetting code
\usepackage{enumitem}
\usepackage{tabularx}

% Define SQL listing style
% \lstset{
%     language=SQL,
%     basicstyle=\ttfamily,
%     keywordstyle=\bfseries,
%     commentstyle=\itshape,
%     showstringspaces=false,
%     numbers=left,
%     numberstyle=\tiny,
%     numbersep=5pt,
%     breaklines=true,
%     frame=single,
%     backgroundcolor=\color{gray!10},
%     captionpos=b
% }
\lstset{
    upquote=true,
    language=SQL,
    showspaces=false,
    basicstyle=\ttfamily,
    keywordstyle=\bfseries\color{blue!60},
    numbers=left,
    numberstyle=\tiny,
    commentstyle=\color{gray},
    backgroundcolor=\color{gray!10}
}

\begin{document}


%% Adding logo
\begin{figure}[h]
		\vspace*{-1em}
		\centering
		\includegraphics[width=0.2\linewidth]{university_logo.png}
		\par
		\vspace*{2em}
		{\Large UNIVERSITY OF CHITTAGONG}
\end{figure}
%% Document information
\begin{center}
		\vspace*{3em}
		\textbf{Department of Computer Science and Engineering} \\
		\bigskip
		Session: 2021-2022 \\
		4th semester \\
		\bigskip
		\begin{tabular}{l l}
		  Assignment No. &: 02\\
		  Course Title &: Database Systems \\
		  Course Code No. &: CSE-413 \\
		\end{tabular}
\end{center}

%% Teacher information
\begin{center}
		\vspace*{3em}
		Submitted to: \\
		\textbf{Dr. Rudra Pratap Deb Nath} \\
		Associate Professor \\
		Department of Computer Science and Engineering \\
		University of Chittagong
\end{center}

%% Student information
\begin{center}
		\vspace*{3em}
		Submitted by: \\
		\textbf{Sanzid Islam Mahi} \\
		ID: 22701065 \\
		Department of Computer Science and Engineering \\
		University of Chittagong
\end{center}




\begin{center}
	\vspace*{3em}
	Date: Jul 06, 2024
\end{center}

\newpage
\section*{Chapter 6}
\subsection*{Practice 6}
\begin{enumerate}
    \item Write a query for the HR department to produce the addresses of all the departments. Use the
LOCATIONS and COUNTRIES tables. Show the location ID, street address, city, state or
province, and country in the output. Use a NATURAL JOIN to produce the results.

    \begin{figure}[h]
        \centering
            \centering
            \includegraphics[width=.7\linewidth]{graphics/61.png}
    \end{figure}
    
    \textbf{Solution: }
    \begin{lstlisting}[language=SQL]
SELECT location_id,
    street_address,
    city,
    state_province,
    country_name
FROM hr.LOCATIONS
NATURAL JOIN hr.COUNTRIES;
    \end{lstlisting}
        \item The HR department needs a report of all employees. Write a query to display the last name,
department number, and department name for all the employees.
    \begin{figure}[h]
        \centering
            \centering
            \includegraphics[width=.5\linewidth]{graphics/62.png}
    \end{figure}
    
    \textbf{Solution: }
    \begin{lstlisting}[language=SQL]
SELECT e.last_name, 
    e.department_id, 
    d.department_name
FROM hr.EMPLOYEES e
LEFT JOIN hr.DEPARTMENTS d 
    ON e.department_id = d.department_id;
    \end{lstlisting}
    %\newpage
    \item The HR department needs a report of employees in Toronto. Display the last name, job,
department number, and the department name for all employees who work in Toronto.
    \begin{figure}[h]
        \centering
            \centering
            \includegraphics[width=.6\linewidth]{graphics/63.png}
    \end{figure}
    
    \textbf{Solution: }
    \begin{lstlisting}[language=SQL]
SELECT last_name,
       job_id,
       department_id,
       department_name
FROM hr.employees
JOIN hr.departments USING (department_id)
JOIN hr.locations USING (location_id)
WHERE city = 'Toronto';
    \end{lstlisting}
        \item Create a report to display employees' last name and employee number along with their
manager's last name and manager number. Label the columns Employee, Emp\#, Manager,
and Mgr\#, respectively. Save your SQL statement as \texttt{lab\_06\_04.sql}. Run the query.
    \begin{figure}[h]
        \centering
            \centering
            \includegraphics[width=.35\linewidth]{graphics/64.png}
    \end{figure}
    %\newpage
    
     \textbf{Solution: }
    \begin{lstlisting}[language=SQL]
SELECT e.last_name AS Employee, 
    e.employee_id AS Emp#, 
    m.last_name AS Manager, 
    m.employee_id AS Mgr#
FROM hr.EMPLOYEES e
JOIN hr.EMPLOYEES m 
    ON e.manager_id = m.employee_id;
    \end{lstlisting}
    \newpage
        \item \texttt{Modify\ lab\_06\_04.sql} to display all employees including King, who has no manager.
Order the results by the employee number. Save your SQL statement as \texttt{lab\_06\_05.sql}.
Run the query in \texttt{lab\_06\_05.sql}.
    
    \begin{figure}[h]
        \centering
            \centering
            \includegraphics[width=.5\linewidth]{graphics/65.png}
    \end{figure}

    \textbf{Solution: }
    \begin{lstlisting}[language=SQL]
SELECT e.last_name AS Employee, 
    e.employee_id AS Emp#, 
    m.last_name AS Manager, 
    m.employee_id AS Mgr#
FROM hr.EMPLOYEES e
LEFT JOIN hr.EMPLOYEES m 
    ON e.manager_id = m.employee_id
ORDER BY e.employee_id;
    \end{lstlisting}
    \newpage
    \item Create a report for the HR department that displays employee last names, department numbers,
and all the employees who work in the same department as a given employee. Give each
column an appropriate label. Save the script to a file named \texttt{lab\_06\_06.sql}.
    \begin{figure}[h]
        \centering
            \centering
            \includegraphics[width=.5\linewidth]{graphics/66.png}
    \end{figure}
    
    \textbf{Solution: }
    \begin{lstlisting}[language=SQL]
SELECT 
    e1.department_id AS Department#, 
    e1.last_name AS Employee, 
    e2.last_name AS Colleague
FROM hr.EMPLOYEES e1
JOIN hr.EMPLOYEES e2 
    ON e1.department_id = e2.department_id
WHERE e1.employee_id != e2.employee_id
ORDER BY e1.department_id, 
    e1.last_name, 
    e2.last_name;
    \end{lstlisting}

        \item The HR department needs a report on job grades and salaries. To familiarize yourself with the
\texttt{JOB\_GRADES} table, first show the structure of the \texttt{JOB\_GRADES} table. Then create a query
that displays the name, job, department name, salary, and grade for all employees.
    
    % \begin{figure}[h]
    %     \centering
    %         \centering
    %         \includegraphics[width=.5\linewidth]{graphics/67.1.png}
    % \end{figure}
    % \begin{figure}[h]
    %     \centering
    %         \centering
    %         \includegraphics[width=.5\linewidth]{graphics/67.2.png}
    % \end{figure}
    \begin{figure}[h]
    \centering
    \begin{subfigure}[b]{0.45\linewidth}
        \centering
        \includegraphics[width=\linewidth]{graphics/67.1.png}
    \end{subfigure}
    \hfill
    \begin{subfigure}[b]{0.5\linewidth}
        \centering
        \includegraphics[width=\linewidth]{graphics/67.2.png}
    \end{subfigure}
\end{figure}
\newpage
    \textbf{Solution: }
    \begin{lstlisting}[language=SQL]
SELECT last_name,
       job_id,
       department_name,
       salary,
       grade_level
FROM employees
JOIN departments USING (department_id)
JOIN job_grades 
    ON (salary BETWEEN lowest_sal AND highest_sal)
ORDER BY salary;
    \end{lstlisting}
    %\newpage
    \item The HR department wants to determine the names of all the employees who were hired after
Davies. Create a query to display the name and hire date of any employee hired after employee
Davies.
    \begin{figure}[h]
        \centering
            \centering
            \includegraphics[width=.5\linewidth]{graphics/68.png}
    \end{figure}
    
    \textbf{Solution: }
    \begin{lstlisting}[language=SQL]
SELECT last_name ,hire_date
FROM hr.EMPLOYEES e
WHERE hire_date > (
        SELECT hire_date 
        FROM hr.EMPLOYEES 
        WHERE last_name = 'Davies'
    );
    \end{lstlisting}
    \item The HR department needs to find the names and hire dates of all the employees who were hired
before their managers, along with their managers' names and hire dates. Save the script to a file
named \texttt{lab\_06\_09.sql}.
\newpage 
\begin{figure}[h]
        \centering
            \centering
            \includegraphics[width=.7\linewidth]{graphics/69.png}
    \end{figure}
    
    \textbf{Solution: }
    \begin{lstlisting}[language=SQL]
SELECT e.last_name,
    e.hire_date,
    m.last_name as last_name_1,
    m.hire_date as hire_date_1
FROM hr.EMPLOYEES e
JOIN hr.EMPLOYEES m 
	ON e.manager_id = m.employee_id
WHERE e.hire_date < m.hire_date;
    \end{lstlisting}
\end{enumerate}

\newpage








%%%%%%%% CHAPTER 7 %%%%%%%%


\section*{Chapter 7}
\subsection*{Practice 7}
\begin{enumerate}
    \item The HR department needs a query that prompts the user for an employee last name. The query
then displays the last name and hire date of any employee in the same department as the
employee whose name they supply (excluding that employee). For example, if the user enters
Zlotkey, find all employees who work with Zlotkey (excluding Zlotkey).

    \begin{figure}[h]
        \centering
            \centering
            \includegraphics[width=.5\linewidth]{graphics/71.png}
    \end{figure}
    
    \textbf{Solution: } Skipped. (sql*plus required.)
    % \begin{lstlisting}[language=SQL]

    % \end{lstlisting}
        \item Create a report that displays the employee number, last name, and salary of all employees who
earn more than the average salary. Sort the results in order of ascending salary.
    \begin{figure}[h]
        \centering
            \centering
            \includegraphics[width=.6\linewidth]{graphics/72.png}
    \end{figure}
    
    \textbf{Solution: }
    \begin{lstlisting}[language=SQL]
SELECT 
    employee_id, 
    last_name, 
    salary
FROM 
    hr.employees
WHERE salary > (SELECT AVG(salary) 
                FROM hr.employees)
order by salary;
    \end{lstlisting}
    %\newpage
    \item Write a query that displays the employee number and last name of all employees who work in a
department with any employee whose last name contains the letter “u.” Save your SQL
statement as \texttt{lab\_07\_03.sql}. Run your query.
    \begin{figure}[h]
        \centering
            \centering
            \includegraphics[width=.5\linewidth]{graphics/73.png}
    \end{figure}
    
    \textbf{Solution: }
    \begin{lstlisting}[language=SQL]
SELECT employee_id,
    last_name
FROM hr.employees
WHERE department_id IN (
   SELECT department_id
   FROM hr.employees
   WHERE last_name LIKE '%u%'
);
    \end{lstlisting}
        \item The HR department needs a report that displays the last name, department number, and job ID
of all employees whose department location ID is 1700.
    \begin{figure}[h]
        \centering
            \centering
            \includegraphics[width=.7\linewidth]{graphics/74.png}
    \end{figure}
    \newpage
    \textbf{Solution: }
    \begin{lstlisting}[language=SQL]
SELECT last_name,
    department_id,
    job_id
FROM hr.employees
WHERE department_id IN (
   SELECT department_id
   FROM departments
   WHERE location_id = 1700
)
ORDER BY department_id;
    \end{lstlisting}
    Modify the query so that the user is prompted for a location ID. Save this to a file named
\texttt{lab\_07\_04.sql}.\\ \\
\textbf{Skipped (sql*plus required.}
    % \newpage
        \item Create a report for HR that displays the last name and salary of every employee who reports to
King.
    
    \begin{figure}[h]
        \centering
            \centering
            \includegraphics[width=.5\linewidth]{graphics/75.png}
    \end{figure}

    \textbf{Solution: }
    \begin{lstlisting}[language=SQL]
SELECT last_name,salary
FROM hr.employees
WHERE manager_id in (
   SELECT manager_id
   FROM hr.employees
   WHERE last_name = 'King'
);
    \end{lstlisting}
        \item Create a report for HR that displays the department number, last name, and job ID for every
employee in the Executive department.

    
    \begin{figure}[h]
        \centering
            \centering
            \includegraphics[width=.6\linewidth]{graphics/76.png}
    \end{figure}
\newpage
    \textbf{Solution: }
    \begin{lstlisting}[language=SQL]
select department_id, last_name, job_id
from hr.employees
where department_id=(
    select department_id
    from hr.departments
    where department_name = 'Executive'
);
    \end{lstlisting}
        \item Modify the query in \texttt{lab\_07\_03.sql} to display the employee number, last name, and salary
of all employees who earn more than the average salary, and who work in a department with
any employee whose last name contains a ``u." Resave \texttt{lab\_07\_03.sql} as
\texttt{lab\_07\_07.sql}. Run the statement in \texttt{lab\_07\_07.sql}.

    
    \begin{figure}[h]
        \centering
            \centering
            \includegraphics[width=.6\linewidth]{graphics/77.png}
    \end{figure}

    \textbf{Solution: }
    \begin{lstlisting}[language=SQL]
SELECT employee_id,
    last_name,
    salary
FROM hr.employees
WHERE salary > (
        SELECT AVG (salary)
        FROM hr.employees
    )
    AND department_id IN (
        SELECT department_id
        FROM hr.employees
        WHERE last_name LIKE '%u%'
    );
    \end{lstlisting}
\end{enumerate}

\end{document}
