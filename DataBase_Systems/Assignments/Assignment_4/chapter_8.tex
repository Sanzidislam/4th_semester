Insert data into the \texttt{MY\_EMPLOYEE} table.
\begin{enumerate}
%%%%%%%Problem 1%%%%%%%%%%%
    \item Run the statement in the \texttt{lab8\_1.sql} script to build the \texttt{MY\_EMPLOYEE} table to be used for the lab.
    
    \textbf{Solution: }
    \begin{lstlisting}[language=SQL]
    \end{lstlisting}
%%%%%%%Problem 2%%%%%%%%%%%
    \item Describe the structure of the \texttt{MY\_EMPLOYEE} table to identify the column names. 
    
    \textbf{Solution: }
    \begin{lstlisting}[language=SQL]
    \end{lstlisting}
%%%%%%%Problem 3%%%%%%%%%%%
    \item  Add the first row of data to the \texttt{MY\_EMPLOYEE} table from the following sample data. Do not list the
columns in the INSERT clause.   
    
    \textbf{Solution: }
    \begin{lstlisting}[language=SQL]
    \end{lstlisting}
%%%%%%%Problem 4%%%%%%%%%%%
    \item  Populate the \texttt{MY\_EMPLOYEE} table with the second row of sample data from the preceding list. This
time, list the columns explicitly in the INSERT clause.  
    
    \textbf{Solution: }
    \begin{lstlisting}[language=SQL]
    \end{lstlisting}
%%%%%%%Problem 5%%%%%%%%%%%
    \item  Confirm your addition to the table.   
    
    \textbf{Solution: }
    \begin{lstlisting}[language=SQL]
    \end{lstlisting}
%%%%%%%Problem 6%%%%%%%%%%%
    \item  Write an INSERT statement in a text file named loademp.sql to load rows into the
\texttt{MY\_EMPLOYEE} table. Concatenate the first letter of the first name and the first seven characters of
the last name to produce the user ID.  
    
    \textbf{Solution: }
    \begin{lstlisting}[language=SQL]
    \end{lstlisting}
%%%%%%%Problem 7%%%%%%%%%%%
    \item  Populate the table with the next two rows of sample data by running the INSERT statement in the
script that you created. 
    
    \textbf{Solution: }
    \begin{lstlisting}[language=SQL]
    \end{lstlisting}
%%%%%%%Problem 8%%%%%%%%%%%
    \item  Confirm your additions to the table.
    
    \textbf{Solution: }
    \begin{lstlisting}[language=SQL]
    \end{lstlisting}
%%%%%%%Problem 9%%%%%%%%%%%
    \item  Make the data additions permanent.
    
    \textbf{Solution: }
    \begin{lstlisting}[language=SQL]
    \end{lstlisting}

Update and delete data in the \texttt{MY\_EMPLOYEE} table.
%%%%%%%Problem 10%%%%%%%%%%%
    \item  Change the last name of employee 3 to Drexler.
    
    \textbf{Solution: }
    \begin{lstlisting}[language=SQL]
    \end{lstlisting}
%%%%%%%Problem 11%%%%%%%%%%%
    \item  Change the salary to 1000 for all employees with a salary less than 900.
    
    \textbf{Solution: }
    \begin{lstlisting}[language=SQL]
    \end{lstlisting}
%%%%%%%Problem 12%%%%%%%%%%%
    \item  Verify your changes to the table.
    
    \textbf{Solution: }
    \begin{lstlisting}[language=SQL]
    \end{lstlisting}
%%%%%%%Problem 13%%%%%%%%%%%
    \item  Delete Betty Dancs from the \texttt{MY\_EMPLOYEE} table.
    
    \textbf{Solution: }
    \begin{lstlisting}[language=SQL]
    \end{lstlisting} 
%%%%%%%Problem 14%%%%%%%%%%%
    \item  Confirm your changes to the table.
    
    \textbf{Solution: }
    \begin{lstlisting}[language=SQL]
    \end{lstlisting} 
%%%%%%%Problem 15%%%%%%%%%%%
    \item  Commit all pending changes.
    
    \textbf{Solution: }
    \begin{lstlisting}[language=SQL]
    \end{lstlisting}  
Control data transaction to the \texttt{MY\_EMPLOYEE} table.
%%%%%%%Problem 16%%%%%%%%%%%
    \item Populate the table with the last row of sample data by modifying the statements in the script that you
created in step 6. Run the statements in the script.
    
    \textbf{Solution: }
    \begin{lstlisting}[language=SQL]
    \end{lstlisting}  
%%%%%%%Problem 17%%%%%%%%%%%
    \item Confirm your addition to the table.
    
    \textbf{Solution: }
    \begin{lstlisting}[language=SQL]
    \end{lstlisting} 
%%%%%%%Problem 18%%%%%%%%%%%
    \item Mark an intermediate point in the processing of the transaction.
    
    \textbf{Solution: }
    \begin{lstlisting}[language=SQL]
    \end{lstlisting}
%%%%%%%Problem 19%%%%%%%%%%%
    \item Empty the entire table.
    
    \textbf{Solution: }
    \begin{lstlisting}[language=SQL]
    \end{lstlisting}
%%%%%%%Problem 20%%%%%%%%%%%
    \item Confirm that the table is empty.
    
    \textbf{Solution: }
    \begin{lstlisting}[language=SQL]
    \end{lstlisting}
%%%%%%%Problem 21%%%%%%%%%%%
    \item Discard the most recent DELETE operation without discarding the earlier INSERT operation.
    
    \textbf{Solution: }
    \begin{lstlisting}[language=SQL]
    \end{lstlisting}
%%%%%%%Problem 22%%%%%%%%%%%
    \item Confirm that the new row is still intact.
    
    \textbf{Solution: }
    \begin{lstlisting}[language=SQL]
    \end{lstlisting}
%%%%%%%Problem 23%%%%%%%%%%%
    \item Make the data addition permanent.
    
    \textbf{Solution: }
    \begin{lstlisting}[language=SQL]
    \end{lstlisting}
\end{enumerate}